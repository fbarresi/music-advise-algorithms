\documentclass[11pt,twoside,a4paper]{article}
\usepackage{german,a4wide,amsmath,amssymb}

% Mann will direkt Umlaute eingeben können statt \"a, \"o, \"u usw.
% Entweder:
%\usepackage[latin1]{inputenc}
% oder:
%\usepackage{umlaut}


% Trennvorschl"age (in {} einfuegen, wenn nicht automatisch getrennt wird:
% z.B. Authen-ti-ka-tions-sys-tem)
\hyphenation{}

\hyphenation{min-des-tens}


%-------------------------- Formatsachen --------------------------%

% Bild-, Tabellenunterschriften veraendern:
% Nummer fett, kleinerer Text fuer Bildunterschrift
\usepackage[bf,small]{caption}

\usepackage{mathpazo}  % -- Palatino als Zeichensatz -- einfach diese
					   % Zeile auskommentieren, falls nicht installiert
%\usepackage{mathptmx}  % -- Times als Zeichensatz

% Zum Unterscheiden von Entwurfs- und endgueltiger Fassung
%\usepackage{draftcopy}
%\draftcopySetGrey{0.90}   %   90% = sehr helles Grau
%\draftcopyName{ENTWURF}{155}   % statt ``DRAFT''
%\draftcopySetScale{1}

%--------------- Zeilen- und Absatzabstaende ----------------------%
\setlength{\parindent}{0em}
\setlength{\parskip}{\medskipamount}    % Abstand zwischen Abs"atzen

% ---------- Umgebungen f"ur Satz/ Lemma, etc. --------------------%
\newtheorem{satz}{Satz}
\newtheorem{nota}{Notation}
\newtheorem{defi}{Definition}
\newtheorem{kons}{Konstruktion}

\newenvironment{notation}{\noindent \textbf{Notation: }}{}
\newenvironment{beweis}{\noindent \textbf{Beweis: }}{}
\newenvironment{anmerkung}{\noindent \textbf{Anmerkung: }}{}
\newenvironment{anmerkungen}{\noindent \textbf{Anmerkungen: }}{}
\newenvironment{beispiel}{\noindent \textbf{Beispiel: }}{}
\newenvironment{beispiele}{\noindent \textbf{Beispiele: }}{}

% URLs und Mailadressen etc. richtig trennen:
\usepackage{url}
% Auch praktisch fuer Mailadressen: \url{blabla@laberlaber.de}

% --------------------- Eigene Befehle fuer math. Mengen ---------%
\newcommand{\N}{{\rm I\!N}}             % die natuerlichen Zahlen
\newcommand{\Z}{\mathbb{Z}}             % fuer ganze Zahlen
\newcommand{\R}{\mathbb{R}}             % die reellen Zahlen
\newcommand{\Prim}{{\rm I\!P}}          % die Primzahlen


% -- Sinnvolle Befehle, um sich selbst Notizen im Text zu machen --%

% "Ungeordnete Gedanken, die noch irgendwo reinsollen":
\newcommand{\kramsubsection}[1][Unsortierte Textfragmente]{%
\subsection*{#1}%
\addcontentsline{toc}{subsection}{#1}%
}

% Randbemerkung:
\newcommand{\bemerkung}[1]{\marginpar{\small\textsl{\textsf{#1}}}}

% "Hier muss noch [weiter-]geschrieben werden" (Baustellensymbol am Rand)
%
% [Damit dieser Befehl funktioniert, muss man natuerlich erstmal
%  das Icon "Baustelle.eps" besorgen!!  Also entweder selbermachen
%  oder downloaden:
%  http://www.net.in.tum.de/teaching/WS04/routing/Baustelle.eps.gz  ]
\newcommand{\baustelle}[1][]{
 \marginpar{%
   \centerline{\includegraphics[scale=0.3]{Baustelle.eps}}
   {\small\textsl{\textsf{\raggedright #1}}}
}}




\begin{document}

\title{Algorithms and approaches to music recommendation}
\author{Federico Barresi \\
  (\texttt{fridolw@in.tum.de})\\[5mm]
  Wissenschaftliches Arbeiten - Aufgabe 7 , \\
  Technische Hochschule Regensburg
}
  
\date{WS\,2020/2021 (Version vom \today)}

\maketitle

\abstract{Dieses Papier behandelt ... (kurze Zusammenfassung des Inhalts)}

\section{Einleitung}
Hier wird eine Motivation f"ur das Papier gegeben sowie der Aufbau kurz 
erl\"autert.  Grunds"atzlich:  Was ist das Problem?  Warum ist es ein
Problem?  Warum ist es interessant, das Problem zu l"osen?

\section{Kapitel}

\subsection{Unterkapitel}

\begin{defi}[Beispieldefinition]
Gegeben ... wird ... definiert als ..
\end{defi}

\section{Kapitel}

Shamir f"uhrt in \cite{sham79} das Konzept der Geheimniszerlegung ein 
(Beispielzitat).

\section{Zusammenfassung}
Zum Schluss nochmal Zusammenfassung, was Papier behandelt hat und 
Schlussfolgerungen, Verweis auf Praxis o.\,"a.


\begin{thebibliography}{12}

\bibitem[HaKT1 98]{HaKT1 98}
        Michael Harkavy, J. D. Tygar, Hiroaki Kikuchi: {\sl Multi-round 
        Anonymous Auction Protocols}; 1st IEEE Workshop on Dependable and 
        Real-Time E-Commerce Systems, 1998.
\bibitem[Sham\_79]{sham79}
        Adi Shamir: {\sl How to Share a Secret}; 
        Communications of the ACM 22/11 (1979), S. 612-613.
\end{thebibliography}

\end{document}




